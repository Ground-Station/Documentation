\chapter[Autoavaliação]{Autoavaliação}
%\addcontentsline{toc}{chapter}{Contextualização}
\label{autoavaliacao}

\begin{itemize}

%    \item \textbf{Nome:Nome}
%    \begin{itemize}
%        \item 
%    \end{itemize}
    
    \item \textbf{Nome: Augusto Moreno Vilarins}
    \begin{itemize}
    \item Refinamento da definição do produto
    \item Levantamento de requisitos
    \item Storytelling
    \item Protótipo de baixa fidelidade
    \item Wireframe
    \item Reuniões de alinhamento com o cliente
    \item Protótipo de média fidelidade
    \item Auxilio na elaboração do documento do Ponto de Controle 2
    \end{itemize}
    
    
    \item \textbf{Nome: Artur Cardoso de Almeida}
    \begin{itemize}
      \item Dimensionamento da estrutura da Maleta 01 - GCS.
      \item Dimensionamento da estrutura da Maleta 02 - Abastecimento.
      \item Criação dos CADS das maletas.
      \item Criação dos desenhos técnicos das estruturas.
      \item Alinhamento sobre a disposição dos componentes da central de controle com o grupo de eletrônica. 
      \item Renderização da maleta GCS e da maleta de abastecimento.
      \item Pesquisa de custo para fabricação da maleta.
      \item Auxílio na escrita do documento do Ponto de Controle 2 .
    \end{itemize}


    \item \textbf{Nome: Diogo Filipe Sens}
    \begin{itemize}
     \item dimensionamento dos atuadores de abertura das válvulas.
     \item Levantamento das características dos materiais escolhidos para a estrutura das maletas.
     \item Pesquisa e escolha dos materiais para revestimentos das maletas.
     \item Simulação por métodos de elementos finitos do choque das maletas em queda de 1m.
     \item Alinhamento do diagrama lógico do sistema de alimentação.
     \item Revisão geral do documento.
    \end{itemize}

    
    \item \textbf{Nome: Douglas Alves Brandão}
    \begin{itemize}
         \item Definição dos requisitos da estrutura utilizada no protótipo.
         \item Definição dos possíveis materiais a serem utilizados na construção da Ground Station.
         \item Auxílio na elaboração dos CADs da maleta. 
         \item Detalhamento das características dos materiais a serem utilizados na construção da Ground Station.
         \item Auxílio na modelagem do sistema de abastecimento. 
         \item Auxílio na simulação do sistema de abastecimento no software Simulink/Matlab.
         \item Definição dos parâmetros do sistema de abastecimento.
         \item Auxílio na elaboração do documento do Ponto de Controle 2.
    \end{itemize}


    \item \textbf{Nome: Gustavo Cavalcante Linhares}
    \begin{itemize}
         \item Gerenciamento e acompanhamento das atividades do grupo de eletrônica 
         \item Detalhamento da solução de hardware da interface do usuário 
         \item Alinhamento com os gerentes sobre as soluções tomadas e sobre o desenvolvimento do projeto
         \item Alinhamento entre eletrônica, sub áreas do projeto e o Stakeholder sobre as soluções que possuem impacto no trabalho em mais de um grupo dentro do projeto
         \item Criação de diagrama esquemático e PCB da central de controle 
         \item Alinhamento sobre a disposição dos componentes da central de controle com o grupo de estrutura
         \item Revisão da documentação do PC2
    \end{itemize}
    
    
    \item \textbf{Nome: Francisco Matheus Fernandes Gomes}
    \begin{itemize}
     \item Diagramas esquemático do circuito dentro do foguete e confecção do projeto de PCB do mesmo.
    \item Criação do diagrama de blocos do abastecimento do projeto e no auxilio do diagrama geral.
    \item Auxílio na edição e revisão do Ponto de Controle 2.
    \item Auxílio na definição da solução e fluxo de projeto do subgrupo da Eletrônica.
   \item Pesquisa das bibliotecas necessárias para implementação do uso da comunicação LoRa
   \item Auxílio na definição da solução de acionamento e controle eletrônico das válvulas internas e externas ao foguete.
    \end{itemize}
    
    
    
    \item \textbf{Nome: Gabriela Alves da Gama}
    \begin{itemize}
    \item Todo o desenho da arquitetura do software
    \item Ajustes no que foi pensado até o PC1
    \item Auxílio dos diagramas no documento
    \item Documentação de toda a arquitetura
    \item Pesquisa sobre o funcionamento do JavaScript a nível de Backend
    \item Pesquisa sobre comunicação entre client e server
    \item Pesquisa de qual melhor arquitetura para o projeto
    \item Planejamento da construção da parte técnica do software
    \item Pesquisa sobre as melhores ferramentas
    \item Decisão sobre as ferramentas que serão usadas no desenvolvimento
    \end{itemize}

    
    \item \textbf{Nome: Isaque Alves de Lima}
    \begin{itemize}
    \item Gerenciamento de riscos do projeto;
    \item Direcionamento nas reuniões;
    \item Apoio no alinhamento entre as áreas do projeto;
    \item Auxílio na edição e revisão do Ponto de Controle 2;
    \item Alinhamento e acompanhamento das atividades do projeto;
    \item Justificativa de não usar Aprendizado de Máquina;
    \item Definição da arquitetura;
    \item Diagrama de representação da arquitetura;
    \item Diagrama de sequência da solução;
    \item Modelagem dos dados (Conceitual e Lógico);
    \item Atualização das metas e restrições da arquitetura;
    \item Revisão das atividades de software.
    \end{itemize}
    
    
    \item \textbf{Nome: João Henrique Egewarth}
    \begin{itemize}
    \item Gerenciamento e acompanhamento das atividades do grupo de software;
    \item Reuniões com clientes;
    \item Auxílio na edição e revisão do Ponto de Controle 2;
    \item Alinhamento e acompanhamento das atividades do projeto;
    \item Refinamento da definição do produto;
    \item Levantamento de requisitos;
    \item Mapa de requisitos;
    \item Wireframe;
    \item Protótipo de média fidelidade;
    \item Revisão das atividades de software;
    \item Argumentação teórica sobre as atividades de produto propostas;
    \item Storytelling;
    \item Criação de arte para apresentação.
    \end{itemize}
    
    \item \textbf{Nome: Luísa Prospero de Carvalho silva}
    \begin{itemize} 
     \item Auxilio na elaboração do documento do Ponto de Controle 2.
     \item Alinhamento da solução com o grupo de eletrônica.
     \item Confecção do diagrama eletro-mecânico do sistema de alimentação.
     \item caracterização técnica dos componentes de abastecimento.
     \item Modelagem do sistema de abastecimento. 
     \item Simulação do sistema de abastecimento no software Simulink/Matlab.
     \item Auxílio na definição dos parâmetros do sistema de abastecimento.
     \item definição da metodologia e dos blocos a serem usados na simulação hidráulica.
     \item Gerenciamento e acompanhamento das atividades do grupo de estrutura e energia.
     \item Alinhamento com os gerentes sobre as soluções tomadas e sobre o desenvolvimento do projeto.
     \item Revisão da documentação do PC2.
    
    
    
   
    \item Definição da solução e fluxo de projeto do subgrupo de estrutura.
   \item Pesquisa das bibliotecas necessárias para implementação da simulação mecanica.
   \item Auxílio na definição da solução de acionamento das válvulas externas ao foguete.
        
   \end{itemize}
    
    \item \textbf{Nome: Milena Martins Magalhães}
    \begin{itemize}
	\item Dimensionamento das baterias para alimentação dos sistemas
	\item Pesquisa de fabricantes e modelos de baterias
	\item Definição das baterias
	\item Pesquisa do carregador de bateria
	\item Confecção do circuito do carregador no Proteus
	\item Simulação do circuito do carregador no Proteus
	\item Dimensionamento dos condutores 
	\item Auxílio na edição e revisão do documento do Ponto de Controle 2 
    \end{itemize}
    
    
    \item \textbf{Nome: Misael de Souza Andrade}
    \begin{itemize}
    \item Criação do Diagrama esquemático do circuito da base de lançamento e confecção do projeto de PCB do mesmo.
    \item Definição dos protocolos de comunicação e pinagem entre os  sensores e microcontroladores.
    \item Pesquisa das bibliotecas necessárias para implementação do uso dos sensores.
    \item Auxílio na definição da solução de acionamento e controle eletrônico das válvulas internas e externas ao foguete.
    \item Auxílio na definição da solução e fluxo de projeto do subgrupo da Eletrônica.
    \item Auxílio na edição e revisão do Ponto de Controle 2.
    \end{itemize}
    
    
    \item \textbf{Nome: Thainá Rodrigues Fernandes}
    \begin{itemize}
    \item Cálculo dos parâmetros do ignitor
	\item Dimensionamento das baterias para alimentação dos sistemas
	\item Pesquisa de fabricantes e modelos de baterias
	\item Definição das baterias
	\item Simulação do circuito do carregador no Proteus
	\item Atualização dos custos dos componentes do sistema de alimentação
	\item Elaboração dos diagramas em blocos dos sistemas de controle e da base
	\item Elaboração dos diagramas unifilares dos sistemas de controle e da base
	\item Auxílio na edição e revisão do documento do Ponto de Controle 2 
    \end{itemize}

    
    
    
\end{itemize}

