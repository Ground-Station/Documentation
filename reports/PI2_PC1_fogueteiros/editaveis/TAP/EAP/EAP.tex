\section{Estrutura Analítica do Projeto}

Com a necessidade de definir os objetivos e as entregas do projeto, foi criada a Estrutura Analítica do Projeto (EAP) na ferramenta \textit{Mindmeister}. O objetivo desse documento é fazer o planejamento de alto nível das atividades e dos objetivos do projeto, divididos em pacotes de entregas conhecidos como Pontos de Controle (PC). Esse documento será utilizado para fazer o planejamento das  \textit{sprints} de cada PC, para auxiliar na comunicação com os  \textit{Stakerholders}, pois ele detalha e define quais atividades e evoluções estão previstas para cada PC.

A EAP foi definida e acordada entre o Gerente Geral, o Diretor de Qualidade e os Diretores Técnicos. Levando em consideração as necessidades dos \textit{Stakerholders} e do contexto da disciplina. Como apresentado na Imagem \ref{fig:EAP}, o projeto está organizado em 3 pontos de controle, também conhecidos como \textit{Releases}, que consistem em pacotes de entrega que agreguem valor ao projeto e atendam às necessidades definidas no Plano de Ensino. Dentro de cada uma das entregas, estão as grandes frentes de desenvolvimento do Projeto, as quais são: Documentação/Viabilidade Técnica, Software, Eletrônica, Estrutura e Energia.

Em cada uma dessas frentes, estão definidas as atividades macro, ou os objetivos a serem alcançados e entregues nessa  \textit{Release}. Vale lembrar que a metodologia principal do projeto é Ágil, logo as atividades e objetivos podem ser alterados de acordo com a necessidade e a evolução do projeto.