\chapter[Autoavaliação]{Autoavaliação}
%\addcontentsline{toc}{chapter}{Contextualização}
\label{autoavaliacao}

\begin{itemize}

%    \item \textbf{Nome:Nome}
%    \begin{itemize}
%        \item 
%    \end{itemize}
    
        \item \textbf{Nome: André Bargas}
    \begin{itemize}
   		 \item Desenvolvimento da lista de é/não é do projeto
   		 \item Desenvolvimento dos Requisitos específicos de software
   		 \item Definição da arquitetura geral de software
         \item Definição das tecnologias utilizadas para desenvolvimento da solução de software
   		 \item Auxilio na definição do micro computador devido a restrições de infraestrutura de software
   		 
    \end{itemize}
    
        \item \textbf{Nome: Augusto Moreno Vilarins}
    \begin{itemize}
        \item  Elaboração do diagrama de causa e efeito.
        \item  Definição dos objetivos geral e específicos do subgrupo de software.
        \item Auxilio na elaboração do artefato de escrita da visão do produto.
        \item Levantamento de requisitos de usabilidade e requisitos não funcionais de software.
        \item Auxílio na elaboração e revisão do documento do Ponto de Controle 1.
    \end{itemize}
    
    
    \item \textbf{Nome: Artur Cardoso de Almeida}
    \begin{itemize}
        \item Elaboração de um esboço inicial para o protótipo da \textit{Ground Station}.
        \item Pesquisa de materiais para a fabricação da maleta \textit{Ground Station}.
        \item Elaboração de um protótipo conceitual da \textit{Ground Station} utilizando a ferramenta CATIA V5R19.
        \item Renderização do projeto conceitual utilizando a ferramenta KeyShot8.
    \end{itemize}


    \item \textbf{Nome: Diogo Filipe Sens}
    \begin{itemize}
        \item Comunicação da equipe com o cliente, incluindo reunião de alinhamento entre ambos;
        \item Apresentação e familiarização da equipe com o problema enfrentado pelo cliente, bem como seu devido contexto;
        \item Orientação do trabalho da equipe no sentido do atendimento das necessidades do cliente;
        \item Elaboração da contextualização, da problematização e da justificativa do projeto;
        \item Elaboração conjunta do escopo do projeto, na parte da estrutura;
        \item Elaboração conjunta da concepção e detalhamento da solução, na parte de sistema de alimentação;
        \item revisão final do documento.
    \end{itemize}

    
    \item \textbf{Nome: Douglas Alves Brandão}
    \begin{itemize}
        \item Pesquisa dos materiais e métodos de fabricação da maleta \textit{Ground Station}.
        \item Elaboração do desenho assistido por computador (CAD) do protótipo.
        \item Estimativa de custo da estrutura da \textit{Ground Station}.
    \end{itemize}


    \item \textbf{Nome: Gustavo Cavalcante Linhares}
    \begin{itemize}
        \item Gerenciamento e acompanhamento das atividades do grupo de eletrônica
        \item Definição da escolha dos componentes da ground station, levando em conta os requisitos do projeto e as necessidades de cada sub área 
        \item Alinhamento com os outros gerentes sobre atividades concluídas e sobre o andamento do projeto
        \item Construção do Business model Canvas
        \item revisão final do documento.
    \end{itemize}
    
    
    \item \textbf{Nome: Francisco Matheus}
    \begin{itemize}
        \item Definição da escolha dos componentes para a telemetria, levando em conta os requisitos do projeto e as necessidades de cada sub área .
        \item Criação do diagrama de blocos do projeto.
        \item Auxílio na edição e revisão do Ponto de Controle 1.
         \item Auxílio na definição da solução e fluxo de projeto do subgrupo da Eletrônica.
    \end{itemize}
    
    
    \item \textbf{Nome: Gabriela Alves da Gama}
    \begin{itemize}
        \item Auxílio na definição dos requisitos funcionais e não-funcionais do projeto.
        \item Elaboração do 5w1h
        \item Construção da Matriz SWOT.
        \item Auxílio na edição e revisão do Ponto de Controle 1.
        \item Definição da Arquitetura
        \item Definição inicial do projeto
        \item Definição do escopo
    \end{itemize}
    
    
    \item \textbf{Nome: Isaque Alves de Lima}
    \begin{itemize}
        \item Definição da metodologia de trabalho;
        \item Definição dos ritos adotados no projeto;
        \item Definição da estrutura do relatório;
        \item Gestão de Riscos;
        \item Definir e organizar as ferramentas de controle das informações pessoais, de custo e gestão do projeto;
        \item Direcionamento nas reuniões;
        \item Levantamento da EAP e cronograma do projeto;   
        \item Auxílio na edição e revisão do Ponto de Controle 1;
        \item Alinhamento e acompanhamento das atividades do projeto;
        \item EAP - Estrutura Analitica do projeto;
        
        
    \end{itemize}
    
    
    \item \textbf{Nome: João Henrique Egewarth}
    \begin{itemize}
        \item Gerenciamento e acompanhamento das atividades do grupo de software;
        \item Preparação de um dinâmica em grupo para entendimento e alinhamento do problema e da solução;
        \item Alinhamento com os outros gerentes sobre a solução e andamento do projeto;
        \item Construção do Business model Canvas;
        \item Auxilio na definição da arquitetura geral de software
        \item Auxilio na definição das tecnologias utilizadas para desenvolvimento da solução de software
        \item Desenvolvimento da problematização do projeto por meio de um fishbone
        \item Definição do objetivo especifico de software
        \item Auxilio na definição do micro computador devido a restrições de infraestrutura de software
        \item Levantamento da EAP de software
        \item Levantamento do cronograma de software
    \end{itemize}
    
    \item \textbf{Nome: Luísa Prospero de Carvalho silva}
    \begin{itemize}
        \item Gerenciamento e acompanhamento das atividades do subgrupo de estrutura e energia.
        \item Definição dos objetivos geral e específicos do subgrupo de estrutura.  
        \item Levantamento de requisitos de estrutura.
        \item Definição dos elementos estruturais da \textit{Ground Station}.
        \item Alinhamento com os outros gerentes sobre atividades concluídas e sobre o andamento do projeto.
        \item Construção do \textit{Business model Canvas}. 
        \item Construção da Matriz SWOT.
        \item Auxílio na definição da estrutura do relatório.
        \item Levantamento da EAP de estrutura e energia.
        \item Levantamento do cronograma de estrutura e energia.
        \item Auxílio na edição e revisão do Ponto de Controle 1. 
        
   \end{itemize}
    
    \item \textbf{Nome: Milena Martins Magalhães}
    \begin{itemize}
   		 \item Definição dos objetivos específicos e requisitos do sistema na área de energia
   		 \item Levantamento do escopo e lista é/não é da área de energia
		\item Avaliação da possibilidade de microgeração
   		 \item Dimensionamento do sistema de alimentação
		\item Definição dos dispositivos de alimentação
		\item Estimativa inicial de custos para o sistema de alimentação
   		\item Auxílio na edição e revisão do Ponto de Controle 1.
    \end{itemize}
    
    
    \item \textbf{Nome: Misael de Souza Andrade}
    \begin{itemize}
        \item Definição da escolha dos sensores do foguete e da balança, levando em conta os requisitos do projeto e as necessidades de cada sub área.
        \item Auxílio na definição dos requisitos funcionais e não-funcionais do projeto.
        \item Auxílio na definição da solução e fluxo de projeto do subgrupo da Eletrônica.
        \item Construção da Matriz SWOT.
        \item Auxílio na edição e revisão do Ponto de Controle 1.
    \end{itemize}
    
    
    \item \textbf{Nome: Thainá Rodrigues Fernandes}
    \begin{itemize}
   		 \item Definição dos objetivos e requisitos do sistema na área de energia
		\item Avaliação da possibilidade de microgeração
   		\item Dimensionamento do sistema de alimentação
		\item Definição dos dispositivos de alimentação
		\item Pesquisa de fabricantes de bateria
		\item Estimativa inicial de custos para o sistema de alimentação
   		\item Auxílio na edição e revisão do Ponto de Controle 1.
    \end{itemize}
    
    
    
\end{itemize}

