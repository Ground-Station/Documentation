\chapter{Considerações finais}

O presente trabalho buscou desenvolver uma solução completa de apoio ao lançamento de um foguete experimental de propulsão híbrida. Os lançamentos de um foguete desse tipo exigem atos preparatórios antes e de acompanhamento durante e depois da ignição do foguete, os quais, por questões de segurança, precisam ser feitos de maneira remota.
\par Assim, foi-se pensado em uma estrutura capaz de executar os comandos e receber as informações necessárias para a execução do ciclo completo de um lançamento.Um vez que o foguete tenha sido colocado na base, e a mangueira de abastecimento tenha sido acoplada, procedimentos que podem ser feitos manualmente pelos membros da equipe de lançamento. o \textit{Rocket Guidance Station} (RGS) é capaz de fazer a abertura das válvulas do sistema de abastecimento, receber os dados de variação do peso que indicam o quanto o tanque está ficando cheio, executar o fechamento das válvulas, o desacoplamento da mangueira e a ignição do motor do foguete, tudo por rádio-frequência a uma distância segura (500m).
\par Além disso, durante o voo, a RGS é capaz de receber os dados de telemetria do foguete, os quais indicam sua posição, altitude e velocidade, informações essenciais para qualificar o voo e proceder sua recuperação após o voo.
\par O projeto foi pensado como uma solução sob medida para as necessidades da Capital Rocket Team, equipe de foguetes da Universidade de Brasília e pioneira no trabalho com propulsão híbrida. Porém, na medida em que mais e mais equipes passam a se interessar mais por essa tecnologia inovadora, a RGS tem o potencial de atender novas demandas do setor, contribuindo para o desenvolvimento das tecnologias aeroespaciais no Brasil e no mundo.