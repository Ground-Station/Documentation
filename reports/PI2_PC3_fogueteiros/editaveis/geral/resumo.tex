\begin{resumo}
A \textit{Rocket Guidance Station} (RGS) é um equipamento criado com a finalidade de tornar mais fácil e seguro o lançamento de foguetes de propulsão híbrida. Para que isto ocorra o projeto foi desenvolvido de forma a ser capaz de realizar o abastecimento e o lançamento do foguete de forma remota além de obter informações relativas ao voo do foguete em tempo real. Com isso o sistema foi desenvolvido de forma a ser de uso intuitivo para o usuário, com duas estruturas principais compactas e portáteis, que dão suporte aos componentes eletroeletrônicos e a interface de usuário. Além disso é uma estação remota capaz de realizar a abertura e fechamento das válvulas que compõem o sistema de alimentação, bem como de executar a ignição do foguete no momento de seu lançamento. É capaz ainda de receber os dados de telemetria do foguete e da base de lançamento antes e durante o voo.


 \vspace{\onelineskip}
    
 \noindent
 \textbf{Palavras-chaves}: Controle e monitoramento. \textit{Ground Station}. Lançamento. Foguete.
\end{resumo}
