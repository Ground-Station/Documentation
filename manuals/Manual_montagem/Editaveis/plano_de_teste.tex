\chapter{Plano de teste}

Para garantir que o sistema se encontrará em perfeito funcionamento após a finalização da construção recomenda-se a realização de testes em cada componente antes de sua utilização. Para isso é necessário utilizar um multímetro.

\section{Sistema de alimentação}

1. Utilizando o multímetro testar a continuidade de:

\begin{itemize}
    \item Todos os cabos.
    \item Todos os plugues e conectores.
    \item Todos os pontos de solda.
\end{itemize}
		
2. Para testar as baterias:

\begin{itemize}
    \item Com as baterias carregadas, colocar o multímetro na escala de 20V de tensão contínua, encostar a ponta de prova vermelha no terminal positivo da bateria e a ponta preta no terminal negativo. No display deve ser mostrada uma tensão de aproximadamente 12V para cada bateria.
\end{itemize}

\section{Carregador}

1. Utilizando o multímetro testar a continuidade de:

\begin{itemize}
    \item Todos os cabos.
    \item Todas as trilhas da PCI.
    \item Todos os plugues e conectores

\end{itemize}

2. Para testar o transformador:

\begin{itemize}
    \item Com o multímetro na escala de continuidade, verificar a continuidade em cada enrolamento.
    \item Com o multímetro na escala de resistência, verificar a resistência em cada enrolamento, o enrolamento de resistência mais elevada deve ser o de entrada da tensão de 110/220V.
 
\end{itemize}

3. Para testar os resistores:

\begin{itemize}
    \item Com o multímetro na escala de resistência, adequada para a resistência nominal de cada resistor, verificar a resistência dos 3 resistores.
\end{itemize}

4. Para testar o capacitor:

\begin{itemize}
    \item Com o multímetro na escala de continuidade, encostar cada ponta de prova a um dos pinos do capacitor, o valor mostrado no display deve aumentar, indicando que o multímetro injetou tensão no capacitor, ao inverter as pontas de prova o valor mostrado no display deve diminuir e em seguida aumentar, indicando a descarga e recarga do capacitor. Se o teste de continuidade mostrar que as pontas estão em curto o capacitor não está adequado para o uso.
\end{itemize}

5. Para testar os diodos:

\begin{itemize}
    \item Com o multímetro na escala de continuidade, encostar cada ponta de prova a um dos pinos do diodo, no sentido direto, a ponta de prova preta no pino negativo e a vermelha no positivo. Se valor mostrado no display for próximo a zero o diodo está em curto, portanto não pode ser utilizado. No sentido inverso o valor exibido no display deve ser 1.
\end{itemize}

6. Para testar o transistor:

\begin{itemize}
    \item Com o multímetro na escala de continuidade, encostar ponta de prova preta no terminal central do transistor (D) e  a ponta de prova vermelha no terminal da direita (S). Se o multímetro medir um valor entre 0,3V e 0,7V no sentido direto e OL. Ou 1—- no sentido reverso, está indicando que o transistor está bom..
\end{itemize}


\section{PCI's}

\begin{itemize}
    \item Colocar a ponta negativa do multímetro no GND da placa e ligar a placa a fonte e testar se a voltagem correta esta chegando em cada ponto essencial da placa.
    \item Testa a continuidade das trilhas.
    \item Testa a continuidade nos pontos de solda, assegurando que não há soldas frias ou mal feitas.
\end{itemize}