\chapter{Cuidados com os equipamentos}

\begin{center}
    \begin{figure}[H]
    \centering
		\includegraphics[scale=1.6]{Figuras/bateria/iconeimportante.png}
	    \label{iconeimportante}
    \end{figure} 
  
    	Informação importante sobre \textbf{CUIDADOS COM OS EQUIPAMENTOS}
 \end{center} 

\subsubsection*{Manuseio da PCB e componentes eletrônicos}

\PAR É de bom grado ter alguns cuidados para manter em boas condições as placas de circuito impresso-PCI que estão sujeitas a vários fatores de risco como:
\begin{itemize}
\item \textbf{Mecânicos:}
\begin{itemize}
\item Vibrações
\item flexões nas PCI 
\item choques mecânicos
\end{itemize}
\end{itemize}

\begin{itemize}
\item \textbf{Ambientais:}
\begin{itemize}
\item Umidade em excesso
\item Contaminantes pelo ar 
\item Excesso de luz solar
\end{itemize}
\end{itemize}

\begin{itemize}
\item \textbf{Eletrostático:}
\begin{itemize}
\item Descargas elétricas produzidas por atrito e contato humano sem devidos cuidados
\end{itemize}
\end{itemize}

\par {\textbf{Alguns cuidados devem ser tomados:}}

\begin{itemize}
\item Evitar tocar em partes metálicas dos componentes e nos conectores e minimizar o manuseio o máximo possível evitando danos mecânicos;
\item É recomendável segurar a placa de forma a não tocar nas suas trilhas preferível que o manuseamento da mesma seja feito de forma que a pessoa segura a placa pelas suas bordas/ laterais;  
\item Nunca flexione a placa ou utilize de muita força ao manuseá-la pode acarretar em rompimento das trilhas,rompimentos de ligações  de encaixe;
\item Nunca molhe a placa de circuito impresso e evite que a mesma entre em contato com ambientes muito úmidos e ser exposta por muito tempo a luz solar;
\end{itemize}

\subsubsection*{Manuseio das superfícies estruturais}

\par A seguir são apresentados os cuidados que deve-se ter com o equipamento estrutural do produto. Leia com atenção para preservar a vida útil do seu equipamento.

\begin{itemize}
    \item Não exponha as maletas a fumaça intensa ou gases, para não danificar o exterior do produto.
    \item Não armazene ou transporte o líquidos propelente dentro das maletas, por ser potencialmente explosivo.
    \item Mantenha as maletas e cases secas, umidade e líquidos podem danificar o revestimento ou infiltrar nas paredes estruturais.
    \item Não aplique cargas de tração nas maletas, elas não foram projetadas para isso.
    \item Não aplique cargas dinâmicas nas maletas, elas não foram projetadas para isso. E podem entrar em ressonância.
    \item Por possuir exterior emborrachado e estrutura de madeira o sistema é inflamável, não submeta-o a altas temperaturas ou ao fogo.
    \item Não cause impactos na estrutura.
    \item Não torça ou deforme o produto.
    \item É de bom grado que o usuário não sente, ou suba nas maletas estruturais.
    \item Não desmonte, modifique ou conserte as maletas sem supervisão de um técnico qualificado para este fim.
    \item Não utilize objetos pontiagudos, líquidos inflamáveis ou limpadores abrasivos.
    \item Não exponha sua estrutura de PLA a acetona, pois ela ira derreter.
    \item A borracha externa pode quebrar em ambientes de baixa umidade, mantenha sua estrutura sempre hidratada.
\end{itemize}
