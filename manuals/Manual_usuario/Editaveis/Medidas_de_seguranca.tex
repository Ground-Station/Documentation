\chapter{Medidas de segurança}

\textbf{Leia essas medidas de segurança antes de utilizar o sistema. Elas contêm informações gerais de segurança. Siga as informações de aviso e de cuidado para evitar ferimentos a você mesmo ou a outras pessoas e para evitar danos ao produto.}

\section*{AVISO}

\begin{center}
 Avisos de segurança nível gravíssimo

\begin{figure}[H]
\centering
\includegraphics[scale = 0.2]{Figuras/aviso.png}
\end{figure}   
\end{center}


\begin{itemize}
    \item Não utilize cabos elétricos ou conectores danificados, nem tomadas desencaixadas ou danificadas, conexões pouco seguras podem causar choques elétricos ou incêndios.
    \item Não toque no sistema, nos cabos, nos conectores, no interruptor ou na tomada com as mãos molhadas ou outras partes do corpo molhadas.
    \item Não puxe os cabos além da extensão deles, isso pode causar choque elétrico ou incêndio.
    \item Não torça, nem danifique os cabos de eletricidade.
    \item Não faça a conexão direta dos pólos positivo e negativo do carregador ou da bateria.
    \item Não utilize o seu sistema ao ar livre durante tempestades.
    \item Não utilize baterias, carregadores, acessórios e equipamentos que não foram projetados para o sistema. Baterias, carregadores e cabos incompatíveis podem causar ferimentos em pessoas ou danos no aparelho.
    \item O uso de baterias ou carregadores genéricos poderá encurtar a vida útil do produto ou causar danos nele. Pode também causar incêndios ou fazer a bateria explodir.
    \item Não libere as válvulas sem garantir a despressurização do sistema.
    \item Não deixe cair nem cause impacto excessivo sobre as maletas e cases, isso pode danificar os equipamentos ou baterias, causar danos ou diminuir sua vida útil.
    \item Evite que os conectores e as extremidade do carregador entrem em contato com materiais condutivos como líquidos, poeira, pós de materiais metálicos e pontas de lápis. Não toque os conectores com ferramentas pontiagudas ou cause algum impacto neles. Você pode criar um curto-circuito ou corroer os terminais, o que pode resultar em explosão ou incêndio.
    \item Não deixe o sistema perto de crianças ou animais.
    \item Não manuseie uma bateria  danificada ou com vazamento
    \item Para o descarte seguro da bateria leia a seção descarte correto do sistema.
\end{itemize}

\section*{ATENÇÃO}

\begin{center}
    Avisos de segurança nível grave

\begin{figure}[H]
\centering
\includegraphics[scale = 0.1]{Figuras/atenção.png}
\end{figure}
\end{center}


\begin{itemize}

\item Atenção ao utilizar os dispositivos perto de outros aparelhos eletrônicos, pois os mesmos emitem um sinal de frequência que pode causar interferência.
\item Evite utilizar o produto perto de um marca-passo, porque poderá causar interferência. 
\item Se faz o uso de um equipamento médico, contate o fabricante do equipamento antes de usá-lo para determinar se o equipamento poderá ou não ser afetado pelas radiofrequências emitidas pela base.
\item Não acione o sistema de abastecimento perto do cilindro de propelente líquido.
\item Observe sempre os regulamentos, instruções e sinais de aviso nos ambientes de lançamento.
\item Se qualquer peça estiver quebrada, apresentar fumaça ou cheiro de queimado, pare de usá-lo imediatamente.
\item Mantenha as maletas e cases secas, umidade e líquidos podem danificar partes ou circuitos elétricos.
\item Não deixe nenhuma parte do produto cair, pois poderá danificar o sistema.
\item Se os pólos da bateria entrarem em contato com objetos metálicos, poderá ocorrer um incêndio.
\item Atenção a ambientes com altas temperaturas e campos magnéticos intensos. Poderá danificar a estrutura ou o sistema eletroeletrônico.
\item Não entre em contato com o sistema caso ele esteja superaquecido, poderá causar queimadoras na pele.
\item Certifique-se de que os equipamentos de instalação manual pelo usuário se encontram devidamente instalados antes do uso.
\item Utilize o sistema apenas para os fins aos quais se destina

\end{itemize}
